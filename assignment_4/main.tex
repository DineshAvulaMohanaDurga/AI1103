\documentclass[journal,12pt,twocolumn]{IEEEtran}

\usepackage{setspace}
\usepackage{gensymb}
\singlespacing
\usepackage[cmex10]{amsmath}

\usepackage{amsthm}

\usepackage{mathrsfs}
\usepackage{txfonts}
\usepackage{stfloats}
\usepackage{bm}
\usepackage{cite}
\usepackage{cases}
\usepackage{subfig}
\usepackage{paralist}
\usepackage{longtable}
\usepackage{multirow}

\usepackage{enumitem}
\usepackage{mathtools}
\usepackage{steinmetz}
\usepackage{tikz}
\usepackage{circuitikz}
\usepackage{verbatim}
\usepackage{tfrupee}
\usepackage[breaklinks=true]{hyperref}
\usepackage{graphicx}
\usepackage{tkz-euclide}

\usetikzlibrary{calc,math}
\usepackage{listings}
    \usepackage{color}                                            %%
    \usepackage{array}                                            %%
    \usepackage{longtable}                                        %%
    \usepackage{calc}                                             %%
    \usepackage{multirow}                                         %%
    \usepackage{hhline}                                           %%
    \usepackage{ifthen}                                           %%
    \usepackage{lscape}     
\usepackage{multicol}
\usepackage{chngcntr}


\usetikzlibrary{calc,math}
\usepackage{listings}
    \usepackage{color}                                            %%
    \usepackage{array}                                            %%
    \usepackage{longtable}                                        %%
    \usepackage{calc}                                             %%
    \usepackage{multirow}                                         %%
    \usepackage{hhline}                                           %%
    \usepackage{ifthen}                                           %%
    \usepackage{lscape}     
\usepackage{multicol}
\usepackage{chngcntr}
\newtheorem{theorem}{Theorem}[section]
\newtheorem{lemma}[theorem]{Lemma}

\DeclareMathOperator*{\Res}{Res}

\renewcommand\thesection{\arabic{section}}
\renewcommand\thesubsection{\thesection.\arabic{subsection}}
\renewcommand\thesubsubsection{\thesubsection.\arabic{subsubsection}}

\renewcommand\thesectiondis{\arabic{section}}
\renewcommand\thesubsectiondis{\thesectiondis.\arabic{subsection}}
\renewcommand\thesubsubsectiondis{\thesubsectiondis.\arabic{subsubsection}}


\hyphenation{op-tical net-works semi-conduc-tor}
\def\inputGnumericTable{}                                 %%

\lstset{
%language=C,
frame=single, 
breaklines=true,
columns=fullflexible
}
\begin{document}

\newcommand{\BEQA}{\begin{eqnarray}}
\newcommand{\EEQA}{\end{eqnarray}}
\newcommand{\define}{\stackrel{\triangle}{=}}
\newcommand{\R}{\mathbb{R}}
\bibliographystyle{IEEEtran}
\raggedbottom
\setlength{\parindent}{0pt}
\providecommand{\mbf}{\mathbf}
\providecommand{\pr}[1]{\ensuremath{\Pr\left(#1\right)}}
\providecommand{\qfunc}[1]{\ensuremath{Q\left(#1\right)}}
\providecommand{\sbrak}[1]{\ensuremath{{}\left[#1\right]}}
\providecommand{\lsbrak}[1]{\ensuremath{{}\left[#1\right.}}
\providecommand{\rsbrak}[1]{\ensuremath{{}\left.#1\right]}}
\providecommand{\brak}[1]{\ensuremath{\left(#1\right)}}
\providecommand{\lbrak}[1]{\ensuremath{\left(#1\right.}}
\providecommand{\rbrak}[1]{\ensuremath{\left.#1\right)}}
\providecommand{\cbrak}[1]{\ensuremath{\left\{#1\right\}}}
\providecommand{\lcbrak}[1]{\ensuremath{\left\{#1\right.}}
\providecommand{\rcbrak}[1]{\ensuremath{\left.#1\right\}}}
\theoremstyle{remark}
\newtheorem{rem}{Remark}
\newcommand{\sgn}{\mathop{\mathrm{sgn}}}
\providecommand{\abs}[1]{\vert#1\vert}
\providecommand{\res}[1]{\Res\displaylimits_{#1}} 
\providecommand{\norm}[1]{\lVert#1\rVert}
%\providecommand{\norm}[1]{\lVert#1\rVert}
\providecommand{\mtx}[1]{\mathbf{#1}}
\providecommand{\mean}[1]{E[ #1 ]}
\providecommand{\fourier}{\overset{\mathcal{F}}{ \rightleftharpoons}}
%\providecommand{\hilbert}{\overset{\mathcal{H}}{ \rightleftharpoons}}
\providecommand{\system}{\overset{\mathcal{H}}{ \longleftrightarrow}}
	%\newcommand{\solution}[2]{\textbf{Solution:}{#1}}
\newcommand{\solution}{\noindent \textbf{Solution: }}
\newcommand{\cosec}{\,\text{cosec}\,}
\providecommand{\dec}[2]{\ensuremath{\overset{#1}{\underset{#2}{\gtrless}}}}
\newcommand{\myvec}[1]{\ensuremath{\begin{pmatrix}#1\end{pmatrix}}}
\newcommand{\mydet}[1]{\ensuremath{\begin{vmatrix}#1\end{vmatrix}}}
\numberwithin{equation}{subsection}
\makeatletter
\@addtoreset{figure}{problem}
\makeatother
\let\StandardTheFigure\thefigure
\let\vec\mathbf
\renewcommand{\thefigure}{\theproblem}
\def\putbox#1#2#3{\makebox[0in][l]{\makebox[#1][l]{}\raisebox{\baselineskip}[0in][0in]{\raisebox{#2}[0in][0in]{#3}}}}
     \def\rightbox#1{\makebox[0in][r]{#1}}
     \def\centbox#1{\makebox[0in]{#1}}
     \def\topbox#1{\raisebox{-\baselineskip}[0in][0in]{#1}}
     \def\midbox#1{\raisebox{-0.5\baselineskip}[0in][0in]{#1}}
\vspace{3cm}
\title{AI1103-Assignment 4}
\author{Name: Avula Mohana Durga Dinesh Reddy , Roll Number: CS20BTECH11005}
\maketitle
\newpage
\bigskip
\renewcommand{\thefigure}{\theenumi}
\renewcommand{\thetable}{\theenumi}
Download all Latex codes from 
%
\begin{lstlisting}
https://github.com/DineshAvulaMohanaDurga/AI1103/blob/main/assignment_4/main.tex
\end{lstlisting}
\section{Question}
\brak{\text{CSIR-UGC-NET June 2015 Q 104}}Let X and Y be random variables with joint cumulative distribution function $F_{XY}\brak{x,y}$. Then which of the following cnditions are sufficient for $\brak{x_0,y_0} \in \R^2$ to be a point of continuity of $F_{XY}$?
\begin{enumerate}
\item $p_{XY}\brak{x=x_0,y=y_0}=0$
\item Either $p_{XY}\brak{x=x_0}=0$ or $p_{XY}\brak{y=y_0}=0$.
\item $p_{XY}\brak{x=x_0}=0$ and $p_{XY}\brak{y=y_0}=0$.
\item $p_{XY}\brak{x=x_0,y\leq y_0}$=0\\
 and $p_{XY}\brak{x\leq x_0,y=y_0}$=0.
\end{enumerate}

\section{Answer}
Let $F_{XY}\brak{x,y}$ be joint cumulative distribution function and $P_{XY}\brak{x,y}$ be joint probability distribution function.\\
\begin{lemma}
$F_{XY}\brak{x,y}$ is continuous at $\brak{x_0,y_0}$ iff all
\begin{enumerate}
\item $p_{X,Y}\brak{x=x_0,y \leq y_0}$
\item $p_{X,Y}\brak{x \leq x_0,y = y_0}$
\item $p_{X,Y}\brak{x=x_0,y = y_0}$
\end{enumerate}
exists and is finite.
\end{lemma}
\begin{proof}
One of the unique properties of cdf is 
\begin{align}
\text{ continuity of cdf} &\Leftrightarrow \text{ differentiability of cdf}.
\end{align}
Lets check the conditions for differentiability of cdf.\\
For cdf to be differentiable at $\brak{x_0,y_0}$
\begin{enumerate}
\item 
\begin{align}
\lim_{h \rightarrow 0} \frac{F_{XY}\brak{x_0+h,y_0}-F_{XY}\brak{x_0,y_0}}{h} & \in \R\nonumber \\
\Rightarrow p_{XY}\brak{x=x_0,y \leq y_0} & \in \R
\end{align}
\item 
\begin{align}
\lim_{k \rightarrow 0} \frac{F_{XY}\brak{x_0,y_0+k}-F_{XY}\brak{x_0,y_0}}{k} &\in \R \nonumber \\
\Rightarrow p_{XY}\brak{x \leq x_),y=y_0} & \in \R 
\end{align}
\item 
\begin{align}
\lim_{t \rightarrow 0} \frac{F_{X,Y}\brak{x_0+u_1 t,y_0 + u_2 t}-F_{X,Y}\brak{x_0,y_0}}{t} &\in \R \nonumber \\
\Rightarrow p_{X,Y}\brak{x=x_0,y=y_0} &\in \R
\end{align}
\end{enumerate}
Equations 2.0.2 ,2.0.3 ,2.0.4 are the conditions for cdf to be differentiable.\\
But using equation 2.0.1 ,2.0.2 ,2.0.3 ,2.0.4 we can say that lemma 2.1 is true.
\end{proof}
\textbf{Now lets verify the options}
\begin{enumerate}
\item Option 1 $\Rightarrow$ 2.1-3 ,but Option 1 $\not \Rightarrow$ 2.1-1 \& 2.1-2 .Hence it fails lemma 2.1.\\
\textbf{Counter-example:-}\\
Let's consider an example similar to dirac-delta function.
\begin{align}
P_{XY}\brak{x,y}&=\infty \hspace{0.15in} \brak{x,y}=\brak{x_0,y_1} , y_1<y_0  \nonumber \\
&=0 \hspace{0.15in} \text{ otherwise}
\end{align}
Here $P_{XY}\brak{x_0,y_0}$ is 0 still $F_{XY}$ is not continuous at $\brak{x_0,y_0}$ parellel to X-axis
\begin{align}
F_{XY}\brak{x,y}&=1 \hspace{0.3in} x>x_0,y>y_1 \nonumber\\
&=0 \hspace{0.3in} otherwise
\end{align}
\textbf{So option 1 is false}
\item Option 2 $\Rightarrow$ 2.1-3 ,either 2.1-1 or 2.1-2 .Hence it fails lemma 2.1.\\
\textbf{Counter-example:-}\\
Lets consider another function similar to dirac delta function
\begin{align}
P_{XY}\brak{x,y}&=\infty \hspace{0.15in} \brak{x,y}=\brak{x_1,y_0} , x_1<x_0  \nonumber \\
&=0 \hspace{0.15in} \text{ otherwise}
\end{align}
Here $p_{XY}\brak{Y=y_0}=0$ but $F_{XY}$ is not continuous at $\brak{x_0,y_0}$.
\begin{align}
F_{XY}\brak{x,y}&=1 \hspace{0.3in} x>x_1,y>y_0 \nonumber\\
&=0 \hspace{0.3in} otherwise
\end{align}
\textbf{So option 2 is also false}
\item Option 3 satisfies lemma 2.1.\\
\textbf{So option 3 is true}
\item Option 4 also satisfies lemma 2.1.\\
\textbf{So option 4 is also true}
\begin{center}
\textbf{Hence correct options are 3,4.}
\end{center}
\end{enumerate}
\end{document}