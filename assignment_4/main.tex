\documentclass[journal,12pt,twocolumn]{IEEEtran}

\usepackage{setspace}
\usepackage{gensymb}
\singlespacing
\usepackage[cmex10]{amsmath}

\usepackage{amsthm}

\usepackage{mathrsfs}
\usepackage{txfonts}
\usepackage{stfloats}
\usepackage{bm}
\usepackage{cite}
\usepackage{cases}
\usepackage{subfig}
\usepackage{paralist}
\usepackage{longtable}
\usepackage{multirow}

\usepackage{enumitem}
\usepackage{mathtools}
\usepackage{steinmetz}
\usepackage{tikz}
\usepackage{circuitikz}
\usepackage{verbatim}
\usepackage{tfrupee}
\usepackage[breaklinks=true]{hyperref}
\usepackage{graphicx}
\usepackage{tkz-euclide}

\usetikzlibrary{calc,math}
\usepackage{listings}
    \usepackage{color}                                            %%
    \usepackage{array}                                            %%
    \usepackage{longtable}                                        %%
    \usepackage{calc}                                             %%
    \usepackage{multirow}                                         %%
    \usepackage{hhline}                                           %%
    \usepackage{ifthen}                                           %%
    \usepackage{lscape}     
\usepackage{multicol}
\usepackage{chngcntr}


\usetikzlibrary{calc,math}
\usepackage{listings}
    \usepackage{color}                                            %%
    \usepackage{array}                                            %%
    \usepackage{longtable}                                        %%
    \usepackage{calc}                                             %%
    \usepackage{multirow}                                         %%
    \usepackage{hhline}                                           %%
    \usepackage{ifthen}                                           %%
    \usepackage{lscape}     
\usepackage{multicol}
\usepackage{chngcntr}

\DeclareMathOperator*{\Res}{Res}

\renewcommand\thesection{\arabic{section}}
\renewcommand\thesubsection{\thesection.\arabic{subsection}}
\renewcommand\thesubsubsection{\thesubsection.\arabic{subsubsection}}

\renewcommand\thesectiondis{\arabic{section}}
\renewcommand\thesubsectiondis{\thesectiondis.\arabic{subsection}}
\renewcommand\thesubsubsectiondis{\thesubsectiondis.\arabic{subsubsection}}


\hyphenation{op-tical net-works semi-conduc-tor}
\def\inputGnumericTable{}                                 %%

\lstset{
%language=C,
frame=single, 
breaklines=true,
columns=fullflexible
}
\begin{document}

\newcommand{\BEQA}{\begin{eqnarray}}
\newcommand{\EEQA}{\end{eqnarray}}
\newcommand{\define}{\stackrel{\triangle}{=}}
\bibliographystyle{IEEEtran}
\raggedbottom
\setlength{\parindent}{0pt}
\providecommand{\mbf}{\mathbf}
\providecommand{\pr}[1]{\ensuremath{\Pr\left(#1\right)}}
\providecommand{\qfunc}[1]{\ensuremath{Q\left(#1\right)}}
\providecommand{\sbrak}[1]{\ensuremath{{}\left[#1\right]}}
\providecommand{\lsbrak}[1]{\ensuremath{{}\left[#1\right.}}
\providecommand{\rsbrak}[1]{\ensuremath{{}\left.#1\right]}}
\providecommand{\brak}[1]{\ensuremath{\left(#1\right)}}
\providecommand{\lbrak}[1]{\ensuremath{\left(#1\right.}}
\providecommand{\rbrak}[1]{\ensuremath{\left.#1\right)}}
\providecommand{\cbrak}[1]{\ensuremath{\left\{#1\right\}}}
\providecommand{\lcbrak}[1]{\ensuremath{\left\{#1\right.}}
\providecommand{\rcbrak}[1]{\ensuremath{\left.#1\right\}}}
\theoremstyle{remark}
\newtheorem{rem}{Remark}
\newcommand{\sgn}{\mathop{\mathrm{sgn}}}
\providecommand{\abs}[1]{\vert#1\vert}
\providecommand{\res}[1]{\Res\displaylimits_{#1}} 
\providecommand{\norm}[1]{\lVert#1\rVert}
%\providecommand{\norm}[1]{\lVert#1\rVert}
\providecommand{\mtx}[1]{\mathbf{#1}}
\providecommand{\mean}[1]{E[ #1 ]}
\providecommand{\fourier}{\overset{\mathcal{F}}{ \rightleftharpoons}}
%\providecommand{\hilbert}{\overset{\mathcal{H}}{ \rightleftharpoons}}
\providecommand{\system}{\overset{\mathcal{H}}{ \longleftrightarrow}}
	%\newcommand{\solution}[2]{\textbf{Solution:}{#1}}
\newcommand{\solution}{\noindent \textbf{Solution: }}
\newcommand{\cosec}{\,\text{cosec}\,}
\providecommand{\dec}[2]{\ensuremath{\overset{#1}{\underset{#2}{\gtrless}}}}
\newcommand{\myvec}[1]{\ensuremath{\begin{pmatrix}#1\end{pmatrix}}}
\newcommand{\mydet}[1]{\ensuremath{\begin{vmatrix}#1\end{vmatrix}}}
\numberwithin{equation}{subsection}
\makeatletter
\@addtoreset{figure}{problem}
\makeatother
\let\StandardTheFigure\thefigure
\let\vec\mathbf
\renewcommand{\thefigure}{\theproblem}
\def\putbox#1#2#3{\makebox[0in][l]{\makebox[#1][l]{}\raisebox{\baselineskip}[0in][0in]{\raisebox{#2}[0in][0in]{#3}}}}
     \def\rightbox#1{\makebox[0in][r]{#1}}
     \def\centbox#1{\makebox[0in]{#1}}
     \def\topbox#1{\raisebox{-\baselineskip}[0in][0in]{#1}}
     \def\midbox#1{\raisebox{-0.5\baselineskip}[0in][0in]{#1}}
\vspace{3cm}
\title{AI1103-Assignment 4}
\author{Name: Avula Mohana Durga Dinesh Reddy , Roll Number: CS20BTECH11005}
\maketitle
\newpage
\bigskip
\renewcommand{\thefigure}{\theenumi}
\renewcommand{\thetable}{\theenumi}
Download all Latex codes from 
%
\begin{lstlisting}
https://github.com/DineshAvulaMohanaDurga/AI1103/blob/main/assignment_4/main.tex
\end{lstlisting}
\section{Question}
\brak{\text{CSIR-UGC-NET June 2015 Q 104}}Let X and Y be random variables with joint cumulative distribution function F\brak{x,y}. Then which of the following cnditions are sufficient for \brak{x,y}$\in R^2$ to be a point of continuity of F?
\begin{enumerate}
\item P\brak{X=x,Y=y}=0
\item Either P\brak{X=x}=0 or P\brak{Y=y}=0.
\item P\brak{X=x}=0 and P\brak{Y=y}=0.
\item P\brak{X=x,Y\leq y}=0 and P\brak{X\leq x,Y=y}=0.
\end{enumerate}

\section{Answer}
Let F\brak{x,y} be joint cumulative distribution function and P\brak{x,y} be joint probability distribution function.\\
From the definitoion of cummulative distribution function 
\begin{align}
pdf &= \frac{d}{dx} cdf \nonumber \\
P\brak{X} &= \frac{d}{dx} F\brak{X}\nonumber\\
&or\nonumber\\
F\brak{x} &= \int_{X=-\infty}^x P\brak{X}dX\\
\Rightarrow F\brak{x,y} &= \int_{Y=-\infty}^y \int_{X=-\infty}^x P\brak{X,Y}dXdY
\end{align}
For F to be continuous at \brak{x,y}\\
\begin{align}
\lim_{k\rightarrow 0}\lim_{h\rightarrow 0} \brak{F\brak{x+h,y+k} - F\brak{x,y}}&=0\\
\lim_{k\rightarrow 0}\lim_{h\rightarrow 0}\int_{Y=-\infty}^{y+k} \int_{X=-\infty}^{x+h} P\brak{X,Y}dXdY \nonumber\\
-\int_{Y=-\infty}^y \int_{X=-\infty}^x P\brak{X,Y}dXdY&=0 
\end{align}
On expanding the limits of integrals we get
\begin{align}
\lim_{k\rightarrow 0}\lim_{h\rightarrow 0} \int_{Y=y}^{y+k} \int_{X=-\infty}^x P\brak{X,Y}dXdY \nonumber \\
+\int_{Y=-\infty}^y \int_{X=x}^{x+k} P\brak{X,Y}dXdY \nonumber \\
+\int_{Y=y}^{y+k} \int_{X=x}^{x+h} P\brak{X,Y}dXdY&=0
\end{align}
Which is zero if P\brak{X,Y} is defined over the integrating region i.e P\brak{X,Y} should be defined over the intergrating region of 
\begin{enumerate}
\item \[\lim_{k\rightarrow 0}\lim_{h\rightarrow 0} \int_{Y=y}^{y+k} \int_{X=-\infty}^x P\brak{X,Y}dXdY\]
\item \[\lim_{k\rightarrow 0}\lim_{h\rightarrow 0} \int_{Y=-\infty}^y \int_{X=x}^{x+k} P\brak{X,Y}dXdY\]
\item \[\lim_{k\rightarrow 0}\lim_{h\rightarrow 0} \int_{Y=y}^{y+k} \int_{X=x}^{x+h} P\brak{X,Y}dXdY\]
\end{enumerate}
The integrating regions tend to be equal to 
\begin{enumerate}
\item $-\infty<X<x,Y=y$  
\item X=x,$-\infty<Y<y$ 
\item point \brak{x,y} 
\end{enumerate}
Let these be regions 1,2,3 respectively\\
$\Rightarrow$ P\brak{X,Y} should be defined over regions 1,2,3. i.e
\begin{enumerate}
\item P$\brak{-\infty<X<x,Y=y}$  
\item P$\brak{X=x,-\infty<Y<y}$ 
\item P$\brak{x,y} $ should be defined
\end{enumerate}
Which is satisfied by options 3,4.
\end{document}